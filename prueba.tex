\documentclass{article}
\usepackage[utf8]{inputenc}
\usepackage{amsmath}
\usepackage[spanish]{babel}
\title{Apuntes de programación lineal 2}
\author{Alicia Areli Zamora López y Brenda Liliana Baustista Resendiz}


\begin{document}
\maketitle
\tableofcontents

\section{Introducción}
\label{sec:introduccion}


La forma estándar de un problema de programación lineal es:
Dados una matriz $A$ y vectores $b,c$, Maximizar $c^Tx$ sujeto a
$Ax\leq b$.

\maketitle
\subsection{Ejemplos}
Un gerente esta planenado como distribuir la producción de dos
productos entre dos maquinas. Para ser manufacturado cada producto
requiere cierto tiempo (en horas) en cada una de las máquinas.
El tiempo requerido esta resumido en la siguiente tabla: 


\begin{tabular}{|c|c|c|}
  \hline
  Producto &A&B \\
  \hline
  maquina 1&1&2\\
  \hline
  maquina 2&1&1\\
  \hline
\end{tabular}


La máquina 1 está disponible 40 horas a la semana y la dos está
disponible 34 horas a la semana.
Si la utilidad al vender los productos A y B es de 2 y 3 pesos por
unidad respectivamente. ¿cuál debe ser la producción semanal que
maximiza la utilidad?, ¿cuál es la utilidad máxima?

RESPUESTA:
La producción semanal que maximiza la utilidad es 28 productos A y 6
productos de tipo B, cuya utilidad máxima es 74 pesos.
(Este problema se resuelve maximizando una ecuacion sujeta a una
restricciones) cacahuate, todos somos cacahuates, cacahuate cacahuate


\maketitle

\subsection{Matriz}


\begin{equation}
  \label{eq:1}
  A=
  \begin{pmatrix}
    0&1&2\\
    3&-1&5\\
    
  \end{pmatrix}
\end{equation}


\end{document}
